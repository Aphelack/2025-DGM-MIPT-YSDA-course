\usepackage[utf8]{inputenc}
\usepackage{graphicx, epsfig}
\usepackage{amsmath,mathrsfs,amsfonts,amssymb}
\usepackage{floatflt}
\usepackage{epic,ecltree}
\usepackage{mathtext}
\usepackage{fancybox}
\usepackage{fancyhdr}
\usepackage{multirow}
\usepackage{enumerate}
\usepackage{epstopdf}
\usepackage{multicol}
\usepackage{algorithm}
\usepackage[noend]{algorithmic}
\usepackage{tikz}
\usepackage{blindtext}
\usepackage{multido}
\usetheme{default}%{Singapore}%{Warsaw}%{Warsaw}%{Darmstadt}
\usecolortheme{default}

\setbeamerfont{title}{size=\Huge}
\setbeamertemplate{footline}[frame number]{}

\setbeamertemplate{section in toc}[sections numbered]

\makeatletter
\newcommand\HUGE{\@setfontsize\Huge{35}{40}}
\makeatother    

\setbeamerfont{title}{size=\HUGE}
\beamertemplatenavigationsymbolsempty

\usetikzlibrary{arrows,shapes,positioning,shadows,trees}

\newcommand\myfootnote[1]{%
  \vspace{-0.5cm}%
  \tikz[remember picture,overlay]
  \draw (current page.south west) +(1in + \oddsidemargin,0.5em)
  node[anchor=south west,inner sep=0pt]{\parbox{\textwidth}{%
      \rlap{\rule{10em}{0.4pt}}\raggedright\scriptsize \textit{#1}}};}

\newcommand\myfootnotewithlink[2]{%
  \vspace{-0.5cm}%
  \tikz[remember picture,overlay]
  \draw (current page.south west) +(1in + \oddsidemargin,0.5em)
  node[anchor=south west,inner sep=0pt]{\parbox{\textwidth}{%
      \rlap{\rule{10em}{0.4pt}}\raggedright\scriptsize\href{#1}{\textit{#2}}}};}

\AtBeginSection[]
      {
      	\begin{frame}{Outline}
      		\tableofcontents[currentsection]
      	\end{frame}
      }
      \AtBeginSubsection[]{
      	\begin{frame}{Outline}
      		\tableofcontents[currentsection,currentsubsection]
      	\end{frame}
}

\newcounter{noscounter} % Используется для nextonslide команды (обнуляется только на новом слайде)
\newcounter{pcounter} % Используется для pause команды (обнуляется после использования eqpause)
\newcounter{diffcounter} % Считает количество pause после формулы

\newcommand{\nextonslide}[1]{%
  \stepcounter{noscounter}% Прибавляем счетчик nextonslide
  \stepcounter{pcounter}% Прибавляем счетчик pause
  \stepcounter{diffcounter}% Прибавляем счетчик diffcounter
  \onslide<\value{noscounter}->{#1}% Отображаем аргумент в скобках на слайде с номером noscounter
}
\newcommand{\resetonslide}{%
    \setcounter{noscounter}{1}% Сбрасываем счетчик nextonslide
    \setcounter{pcounter}{1}% Сбрасываем счетчик pause
    \setcounter{diffcounter}{0}% Сбрасываем счетчик diffcounter
}

\newcommand{\eqpause}{%
  \multido{\i=1+1}{\value{pcounter}}{\pause}% Повторяем pcounter раз команду pause
  \stepcounter{noscounter}% Прибавляем счетчик nextonslide
  \setcounter{pcounter}{1}% Сбрасываем счетчик pause
}

\newcommand{\eqpausediff}{% Вспомогательная команда, запускается автоматически после формул
  \multido{\i=1+1}{\value{diffcounter}}{\pause}% Повторяем diffcounter раз команду pause
  \addtocounter{pcounter}{-\value{diffcounter}}% Вычитаем из pcounter количество сделанных pause
  \setcounter{diffcounter}{0}% Сбрасываем счетчик diffcounter
}

\newcommand\AtEndBoth[2]{% Применяем команду к multline и multline*
  \AtEndEnvironment{#1}{#2}%
  \AtEndEnvironment{#1*}{#2}%
}

\AtEndBoth{align}{\eqpausediff}
\AtEndBoth{equation}{\eqpausediff}
\AtEndBoth{multline}{\eqpausediff}

\addtobeamertemplate{frametitle}{\resetonslide}{}% На каждом слайде сбрасываем счетчики

% latin bold lower
\newcommand{\ba}{\mathbf{a}} 
\newcommand{\bc}{\mathbf{c}} 
\newcommand{\be}{\mathbf{e}} 
\newcommand{\bff}{\mathbf{f}} % \bf - for bold type
\newcommand{\bg}{\mathbf{g}} 
\newcommand{\bh}{\mathbf{h}} 
\newcommand{\bp}{\mathbf{p}} 
\newcommand{\bq}{\mathbf{q}} 
\newcommand{\bt}{\mathbf{t}} 
\newcommand{\bs}{\mathbf{s}} 
\newcommand{\bu}{\mathbf{u}} 
\newcommand{\bv}{\mathbf{v}} 
\newcommand{\bw}{\mathbf{w}} 
\newcommand{\bx}{\mathbf{x}} 
\newcommand{\by}{\mathbf{y}} 
\newcommand{\bz}{\mathbf{z}} 

% latin bold upper
\newcommand{\bA}{\mathbf{A}} 
\newcommand{\bB}{\mathbf{B}} 
\newcommand{\bC}{\mathbf{C}} 
\newcommand{\bG}{\mathbf{G}} 
\newcommand{\bI}{\mathbf{I}} 
\newcommand{\bJ}{\mathbf{J}} 
\newcommand{\bL}{\mathbf{L}} 
\newcommand{\bM}{\mathbf{M}} 
\newcommand{\bP}{\mathbf{P}}
\newcommand{\bQ}{\mathbf{Q}} 
\newcommand{\bR}{\mathbf{R}} 
\newcommand{\bT}{\mathbf{T}} 
\newcommand{\bU}{\mathbf{U}} 
\newcommand{\bV}{\mathbf{V}} 
\newcommand{\bW}{\mathbf{W}} 
\newcommand{\bX}{\mathbf{X}} 
\newcommand{\bY}{\mathbf{Y}} 
\newcommand{\bZ}{\mathbf{Z}} 

% latin cal upper
\newcommand{\cF}{\mathcal{F}} 
\newcommand{\cG}{\mathcal{G}} 
\newcommand{\cI}{\mathcal{I}} 
\newcommand{\cL}{\mathcal{L}} 
\newcommand{\cM}{\mathcal{M}} 
\newcommand{\cN}{\mathcal{N}} 
\newcommand{\cP}{\mathcal{P}} 
\newcommand{\cS}{\mathcal{S}} 
\newcommand{\cT}{\mathcal{T}} 
\newcommand{\cW}{\mathcal{W}} 
\newcommand{\cX}{\mathcal{X}} 
\newcommand{\cZ}{\mathcal{Z}} 

% latin bb upper
\newcommand{\bbE}{\mathbb{E}} 
\newcommand{\bbI}{\mathbb{I}} 
\newcommand{\bbP}{\mathbb{P}} 
\newcommand{\bbR}{\mathbb{R}} 

% greek bold lower
\newcommand{\bepsilon}{\boldsymbol{\epsilon}} 
\newcommand{\btheta}{\boldsymbol{\theta}} 
\newcommand{\blambda}{\boldsymbol{\lambda}} 
\newcommand{\bpi}{\boldsymbol{\pi}} 
\newcommand{\bmu}{\boldsymbol{\mu}} 
\newcommand{\bsigma}{\boldsymbol{\sigma}} 
\newcommand{\bphi}{\boldsymbol{\phi}} 

% greek bold upper
\newcommand{\bSigma}{\boldsymbol{\Sigma}} 

\DeclareMathOperator*{\argmin}{arg\,min}
\DeclareMathOperator*{\argmax}{arg\,max}
\newcommand{\createdgmtitle}[1]{\title[\hbox to 56mm{Deep Generative Models  \hfill\insertframenumber\,/\,\inserttotalframenumber}]
	{\vspace{1cm} \\ \textbf{Deep Generative Models} \\ {\Huge Lecture #1}}
	\author{Roman Isachenko}
	\institute{
		Moscow Institute of Physics and Technology \\
		Yandex School of Data Analysis
	}
	\date{2025, Autumn}
}